\documentclass[a4paper, oneside, final]{scrartcl}

\usepackage[utf8]{inputenc}
\usepackage[brazil]{babel}
\usepackage{amssymb}

\usepackage{soul}
\usepackage{scrpage2}
\usepackage{titlesec}
\usepackage{marvosym}
\usepackage{tabularx}
\usepackage{textcomp}
\newcommand{\vspc}{\vspace{0.15cm}} %espaçamento vertical
\newcommand{\vspcitem}{\vspace{0.1cm}} %espaçamento vertical

\usepackage[hmargin=1.0cm,vmargin=1.0cm,noheadfoot]{geometry}

\titleformat{\section}{\large\scshape\raggedright}{}{0em}{}[\titlerule]
\pagestyle{scrheadings}
\renewcommand{\headfont}{\normalfont\rmfamily\scshape}

\begin{document}

\begin{center}
\textsc{\Huge{Luiz Henrique Leme}} \vspc\\
{\small Brasileiro, +55 (11) 97513-2268, luizhleme@outlook.com}

\section{Objetivo}
	{\large \bf Posição de Liderança na área de Tecnologia da Informação.}
	
\section{Resumo}

\begin{tabularx}{1\linewidth}{X}
	$\star$ Carreira em desenvolvimento na área de Tecnologia da Informação, tendo atuado em empresas multinacionais e nacionais de médio e grande porte. \vspc\\
	
	$\star$ Engenheiro de software com mais de 10 anos de experiência em progrmação e com entusiasmo por software e novas tecnologias. \vspc\\
	
	$\star$ Liderança ágil com Scrum, Kanabn, times próprios, com foco no desenvolvimento de equipes de alta performance e cumprimento dos objetivos. \vspc\\
	
	$\star$ Excelentes conhecomentos em diferentes linguagens de programação e tecnologias, incluindo Java, .Net, C, Cobol, Android, tecnologias web e SQL. \vspc\\	
\end{tabularx}


\section{Idiomas}

\begin{tabularx}{1\linewidth}{p{5cm}X}
Português  & Nativo \\
Espanhol   & Intermediário \\  
Inglês     & Avançado\\
\end{tabularx}

\section{Formação Acadêmica}

\begin{tabularx}{1\linewidth}{p{2cm}X}
2012$-$2015 & Mestrado em Engenharia da Computação\\
            & Instituto de Pesquisas Tecnológicas, IPT\\
            & Título:Uma Ferramenta para a Automatização de Testes de Aceitação Baseada na Linguagem LETAE e no SBVR \\
            & Área: Engenharia de Software \vspc\\
\end{tabularx}

\begin{tabularx}{1\linewidth}{p{2cm}X}
2010$-$2011 & Pós-Graduação em Engenharia de Software\\
            & Instituto Brasileiro de Tecnologia Avançada, IBTA\\
            & Título: Estudo de Caso de uma Empresa Nacional de Médio Porte do Segmento de T.I. Melhoria de Processo de Desenvolvimento de Software\\
            & Área: Engenharia de Software \vspc\\
\end{tabularx}

\begin{tabularx}{1\linewidth}{p{2cm}X}
2006$-$2010 & Bacharelado em Sistemas de Informação\\
            & Universidade Paulista, UNIP \vspc\\
\end{tabularx}

\begin{tabularx}{1\linewidth}{p{2cm}X}
2005$-$2006 & Técnico em Informática\\
            & Escola de Ensino Infantil e Fundamental e Educação Profissional de Jovens e Adultos Embaixador “Assis Chateaubriand”, Fundação Bradesco \vspc\\
\end{tabularx}


\section{Formação Complementar}
\begin{tabularx}{1\linewidth}{p{2cm}X}
2016       & Bitcoin (8h) - FIAP \vspcitem\\
\end{tabularx}

\begin{tabularx}{1\linewidth}{p{2cm}X}
2015       & Web com ASP.NET MVC e Persistência com NHibernate (40h) - Caelum \vspcitem\\
\end{tabularx}

\begin{tabularx}{1\linewidth}{p{2cm}X}
2015       & C\# e Orientação a Objetos (40h) - Caelum \vspcitem\\
\end{tabularx}

\begin{tabularx}{1\linewidth}{p{2cm}X}
2014       & Programação Front-end com Javascript e jQuery (32h) - Caelum \vspcitem\\
\end{tabularx}

\begin{tabularx}{1\linewidth}{p{2cm}X}
2013       & Arquitetura e Design de Projetos Java (40h) - Caelum \vspcitem\\
\end{tabularx}

\begin{tabularx}{1\linewidth}{p{2cm}X}
2012       & Java EE Avançado e Web Services (40h) - Caelum \vspcitem\\
\end{tabularx}

\begin{tabularx}{1\linewidth}{p{2cm}X}
2012       & Web Rica com JSF2, Primeface 4, CDI e JBoss Seam (32h) - Caelum \vspcitem\\
\end{tabularx}

\begin{tabularx}{1\linewidth}{p{2cm}X}
2012       & Persistência com JPA, Hibernate (40h) - Caelum \vspcitem\\
\end{tabularx}

\begin{tabularx}{1\linewidth}{p{2cm}X}
2012       & Laboratório de Java com Testes, XML e Desing Patterns (20h) - Caelum \vspcitem\\
\end{tabularx}

\begin{tabularx}{1\linewidth}{p{2cm}X}
2010       & Java para Desenvolvimento Web (40h) - Caelum \vspcitem\\
\end{tabularx}

\begin{tabularx}{1\linewidth}{p{2cm}X}
2010       & Java e Orientação a Objetos (40h) - Caelum \vspcitem\\
\end{tabularx}

\begin{tabularx}{1\linewidth}{p{2cm}X}
2011       & Formação Programador Mainframe (160h) - Dedic GPTI \vspcitem\\
\end{tabularx}

\begin{tabularx}{1\linewidth}{p{2cm}X}
2010       & Gerenciamento Ágil de Projetos de Software com Scrum (20h) - Caelum
\end{tabularx}

\section{Experiência Profissional}
\begin{tabularx}{1\linewidth}{X}
{\bf CCEE $\star$ Analista Desenvolvedor \hfill Jan/2016 $-$ Atual} \\
Atuando como desenvolvedor e analista de sistema do Sistema de Coleta de Dados de Energia (SCDE) e sustentação de outras aplicações. \vspc\\
\end{tabularx}

\begin{tabularx}{1\linewidth}{X}
{\bf BRQ $\star$ Analista Desenvolvedor \hfill Set/2013 $-$ Dez/2015} \\
Desenvolvendo e melhorando sistemas web (CliqCCEE, SIGACCEE) desenvolvidos em Java na Câmara de Comercialização de Energia Elétrica. \vspc\\
Gestor de Configuração de Software código e documentação dos sistemas da CCEE. \vspc\\
\end{tabularx}

\begin{tabularx}{1\linewidth}{X}
{\bf InfoSERVER S/A $\star$ Analista Desenvolvedor \hfill Mai/2011 $-$ Set/2013} \\
Atuando diretamente na análise, conversão e desenvolvimento das transações contábeis (Pagamento de
Boleto de Cobrança, Depósitos e Saques) do Sistema de Caixa de Agências para o Terminal Financeiro (Sistema
Desktop em Java) do Banco Bradesco.\vspc\\
Responsável pela preparação do ambiente de desenvolvimento FDDB e Terminal Financeiro para a equipe da
InfoSERVER. \vspc\\
Desenvolvimento e manutenção do TineTypes (DomainTypes e ViewWrapper) do Terminal Financeiro.  \vspc\\
\end{tabularx}

\begin{tabularx}{1\linewidth}{X}
{\bf Stefanini IT Solutions $\star$ Analista de Sistemas\hfill Nov/2010 $-$ Abr/2011} \\
Líder técnico da equipe Java Web do projeto UORG do Banco Bradesco. Atuando diretamente com o time
interno de COBOL bem como com as empresas terceiras de COBOL, para validação dos programas. Atuando
com o desenvolvimento e redesenho da aplicação usando a Arquitetura Web Bradesco (AWB).\vspc\\
\end{tabularx}

\begin{tabularx}{1\linewidth}{X}
{\bf InfoSERVER S/A $\star$ Desenvolvedor \hfill Nov/2012 $-$ Out/2010} \\
Atuando com desenvolvimento Web Java usando a Arquitetura Web Bradesco – AWB e Desktop Java com o
FDDB para o Terminal Financeiro. \vspc\\
Líder técnico da equipe de automatização de testes de aceitação utilizando ferramenta IBM Rational
Robot. Atuando com desenvolvimento desktop com Visual Basic 6.0 e VBA. \vspc\\
Principais realizações: Conversão das transações não contábeis do Sistema de Caixa de Agências para o
Terminal Financeiro para o Branco Bradesco S.A.; Desenvolvimento do Sistema Corporate DOC/TED e
Workflow Contratos para o Banco Bradesco S.A.; Elaboração do material de treinamento sobre as ferramentas
IBM Rational Robot e TestManager para o Banco Bradesco S.A.; Desenvolvimento da infraestrutura para
automatização dos testes de aceitação do Sistema de Caixa do Banco Bradesco S.A. usando a ferramenta IBM
Rational Robot; Pré-venda de projeto de automatização de testes de aceitação usando IBM Rational Robot para o Banco Santander; Automatização dos testes de desempenho do sistema CidadeTran do Banco Bradesco 
S.A.; Desenvolvimento do sistema Visas SMS para Banco Bradesco S.A.; Desenvolvimento do sistema de
Bloqueio de Cartões para o Banco Bradesco S.A.; Desenvolvimento do sistema APESM para o MIS Prime do
Banco Bradesco S.A.; 

\end{tabularx}

\section{Trabalhos Voluntários}
\begin{tabularx}{1\linewidth}{X}
{\bf Doutor Palhaço na ONG Presente de Alegria \hfill Out/2015 - Atual} \\
Como Doutor Palhaço visito instuições (Asilos, Hospitais e Orfanatos). \vspc\\
\end{tabularx}

\end{center}

\end{document}